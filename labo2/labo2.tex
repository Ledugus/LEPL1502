\documentclass{exam}
\usepackage[a4paper,left=1.5cm,right=1.5cm,top=1.5cm,bottom=1.5cm, footskip=0.5cm]{geometry}
% Package langue
\usepackage[utf8]{inputenc}
\usepackage[T1]{fontenc}
\usepackage[french]{babel}

%Packages mathématiques
\usepackage{amsmath, amsfonts}
\usepackage{amssymb,amsmath}
\usepackage{amsfonts}
\usepackage{amsthm}


\usepackage{framed}
\usepackage{tabularx}
% Header
\extraheadheight{.95in}
\header{UCLouvain -- EPL \\
        LEPL1502: \textsc{Projet 2}\\
        Laboratoire  -- Q2 2023\\
        [1pt]}{}{  Groupe: \fbox{\parbox[c][8mm][c]{50mm}{\centering 11.55}} \\ 
         } 
% Footer
\usepackage{lastpage}
\cfoot{\thepage /\ \pageref{LastPage}}

%Packages Figures et graphiques
\usepackage{graphicx} %inclusion de figures
\usepackage{pgf,tikz}
\usetikzlibrary{babel} %%% <--- Don't forget
\usepackage[siunitx,american, RPvoltages]{circuitikz}
\usepackage{caption}

\begin{document}

\begin{center}
	\textbf{\huge Laboratoire 2 : Amplificateur opérationnel}
\end{center}

\section*{\textbf{\large Expérience 1 : Comparateur de tension }}
Objectif de l'expérience : Le but de cette expérience est d’observer la tension de sortie $V_{out}$ du montage en comparateur de tension qui est dépendante de la différence entre les tensions d’entrées $V_{in+}$ et $V_{in-}$.
\newline
\begin{minipage}[t]{1\textwidth}
	\begin{minipage}{0.5\textwidth}
		\begin{framed}
			\begin{minipage}[t][7cm][t]{1\textwidth}
				\textbf{Circuits:}\\
				\begin{circuitikz}
					\ctikzset{bipoles/length=1cm}
					\draw (0,0) node[op amp](A1){}
					(A1.+) to [vsourcetri] ++(0, -2) coordinate(ground) to (ground-|A1.out) to [R] (A1.out)
					(A1.down) to (A1.down|-ground)
					(ground) node[ground]{} to ++(-2, 0) to [V=$DC$] ++(0, 4) coordinate(tmp) to (tmp-|A1.up) to (A1.up)
					;
				\end{circuitikz}
			\end{minipage}
		\end{framed}
	\end{minipage}
	\begin{minipage}{0.5\textwidth}
		\begin{framed}
			\begin{minipage}[t][7cm][t]{1\textwidth}
				\textbf{Équations et/ou courbes théoriques:}
				\begin{align}
					V_{th} = & V_0                                   \\
					R_{th} = & R_1 \left(\frac{V_{th}}{V_1}-1\right)
				\end{align}
			\end{minipage}
		\end{framed}
	\end{minipage}
\end{minipage}

\begin{framed}
	\begin{minipage}[t]{0.9\textwidth}
		\textbf{Mode opératoire :}
		Faire les mesures suivantes avec un signal carré de fréquence 1kHz et d’amplitude 1V avec un
		offset de 1V (c-a-d un signal qui alterne entre 0 et 2 V).
		\begin{enumerate}
			\item Mesurer la tension $V_0$ dans le circuit représenté à la Figure 3a (circuit ouvert).
			\item Mesurer la tension $V_1$ dans le circuit représenté à la Figure 3b pour quatre valeurs de $R_1$ entre
			      200Ω et 800Ω.
			\item Déterminer les valeur de $V_{th}$ et $R_{th}$ sur base de la mesure commune de $V_0$ et des différentes
			      mesures de $V_1$ et de votre expression analytique.
			\item Faire varier la fréquence du signal carré (vous pouvez garder la dernière résistance utilisée) et
			      observer l’influence sur les mesures et les résultats.
			\item Faire varier l’amplitude du signal carré et observer l’influence sur les mesures et les résultats.
		\end{enumerate}
	\end{minipage}
\end{framed}
\begin{framed}
	\begin{minipage}[t]{0.9\textwidth}
		\textbf{Observations :}\\
		Pour un potentiel de $1V$ imposé (variation de $R_1$) :\\
		\begin{center}
			\begin{tabularx}{0.6\textwidth} {
					| >{\centering\arraybackslash}X
					| >{\centering\arraybackslash}X
					| >{\centering\arraybackslash}X|}
				\hline
				$R_1~(\Omega)$ & $V_1~(mV)$ & $R_{th}$ déduit ($\Omega$) \\
				\hline
				200            & 245        & 616                        \\
				\hline
				400            & 395        & 613                        \\
				\hline
				600            & 505        & 588                        \\
				\hline
				800            & 575        & 591                        \\
				\hline
			\end{tabularx}\\
		\end{center}
		Pour une résistance de $800\Omega$ imposée (variation de $V_{th}$):\\
		\begin{center}
			\begin{tabularx}{0.6\textwidth} {
					| >{\centering\arraybackslash}X
					| >{\centering\arraybackslash}X
					| >{\centering\arraybackslash}X|}
				\hline
				$V_0~(mV)$ & $V_1~(mV)$ & $R_{th}$ déduit ($\Omega$) \\
				\hline
				1000       & 564        & 618                        \\
				\hline
				800        & 453        & 612                        \\
				\hline
				600        & 339        & 613                        \\
				\hline
				400        & 224        & 628                        \\
				\hline
			\end{tabularx}\\
		\end{center}
		Pour des variations de fréquence, on n'observe pas de changement dans les tensions mesurées.

	\end{minipage}
\end{framed}
\begin{framed}
	\begin{minipage}[t]{\textwidth}
		\textbf{Résultats : }
		La résistance interne du picoscope vaut environ 600 $\Omega$.
	\end{minipage}
\end{framed}
\end{document}
